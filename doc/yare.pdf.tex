\documentclass[11pt]{article}
\usepackage[spanish,activeacute]{babel}
%Gummi|061|=)
\title{\textbf{Lenguaje de programaci\'on Yare 0.0.1}}
\author{Leonardo Gutiérrez Ramírez\\
		\\
		Instituto Tecnológico de Chihuahua II}
\date{17 de Junio del 2012}
\setcounter{page}{1}
\begin{document}

\maketitle
\section{?`Por qu\'e otro lenguaje de programaci\'on?}

Mi gusto por la programaci\'on comenz\'o con la lectura de algunos art\'iculos que curioseando en Internet me llegu\'e a topar.\\
\\Dichos art\'iculos trataban sobre un "lenguaje" que interpretaba archivos en donde se pon\'ia una serie de c\'odigos 'extra\~nos' y la
m\'aquina interpertaba, este lenguaje es {\bf Batch}. Un lenguaje arcaico bastante limitado (con tan solo decir que no tiene soporte para 
n\'umeros decimales) pero en el que era o es divertido para los principiantes realizar peque\~nos c\'odigos y tener resultados sobre el Sistema Operativo
relativamente r\'apidos. Programando en Batch me divert\'ia y esto cultiv\'o en m\'i un amor hac\'ia la programaci\'on. Sin embargo, siempre
me sent\'i limitado por dicho lenguaje, por eso he decidido intentar crear mi propio lenguaje de programaci\'on.\\

No espero competir con los dem\'as lenguajes de programaci\'on, ser\'ia una tonter\'ia, lo que s\'i espero es que los que est\'en intentando
crear algo parecido a un lenguaje, encuentren un camino m\'as f\'acil con el que yo me top\'e. Es por eso que el c\'odigo de {\bf Yare} est\'a
disponible para modificarlo y claro si ustedes gustan contribuir, pues !`adelante!\\

Programar un lenguaje es una tarea bastante dif\'icil(por lo menos as\'i me ha parecido) y agotadora, requiere de an\'alisis y sobre todo de 
mucho tiempo. Sin embargo, pienso que es una de las tareas que como programador te deja m\'as experiencias.

\section{Sobre los errores ...}
No dudo que el c\'odigo contenga errores y d\'e resultados inesperados, por lo que agradecer\'ia que me lo hicieran saber por medio de mi
correo electr\'onico.

{\bf leorocko13@hotmail.com} \'o {\bf leogutierrezramirez@gmail.com}\\

\newpage
\begin{center}{\bf \LARGE Caracter\'isticas del lenguaje}\end{center}
\section{Tipos de datos}

Por ahora no hay varios tipos de datos, el \'unico tipo de dato soportado es {\bf double} de $(8 bytes )$
\linebreak {\bf A}unque se asignen variables con un tipo entero:\\
$x = -4536;$ /* Omitiendo el punto decimal */ \\
{\bf E}sta se asigna internamente como:\\
$x = -4536{.}0000;$ .\\

\section{Variables y su declaraci\'on}

{\bf Yare} posee por defecto $26$ variables, estas son las variables $a\ ...\ z$, las cuales tambi\'en son de tipo $double$.\\
Estas variables pueden utilizarse en cualquier parte del programa, siempre estar\'an disponibles.

Es indistinto utilizar la variable '$a$' \ o '$A$', estas variables no son sensibles a may\'usculas y min\'usculas.\\

$a = 1 + 2 * 3;$\\
$a = b;$\\
$A = z + c * 4.5;$\\
$A = A - 1.0$\\

{\bf Declaraci\'on de variables}\\

Los identificadores de las variables deben estar delimitados por los caracteres '$:$'\ de la siguiente manera:\\

${\bf :mi\ variable:\ } =\ 23.235;$\\
${\bf :mi\ variable:\ } =\ {\bf:ok:}\ *\ -5.67;$\\
${\bf : +-\:\ .\_01293 :}\ =\ {\bf:ok:}\ *\ -5.67;$

Los identificadores pueden ser lo suficiente flexibles y pueden ser de cualquier longitud.\\

{\bf O}tra manera de declarar variables es utilizar la sentencia $declare()$ de la siguiente manera:\\
{\bf $declare(:id:, expr)$}\\

{\bf Ejemplos:}\\

{\bf $declare(:resultados:,\ 1\ +\ 2\ +\ 3);$}\\
{\bf $declare(:resultados:);$}\\

El {\bf primer} ejemplo inicializa la variable '${\bf resultado}$' al valor de la expresión especificada después de la coma.

El {\bf segundo} ejemplo inicializa la variable '${\bf resultado}$'\ a $0$.
\newpage


Support for two high-level {\LaTeX} building systems, \emph{rubber}\footnote{https://launchpad.net/rubber/} \& \emph{latexmk}\footnote{http://www.phys.psu.edu/{\textasciitilde}collins/software/latexmk-jcc/} has been added to this release as well. Your preferred typesetter can be configured through the Compilation tab in the Preferences menu. Typesetters that are not installed on your system will not be selectable. 

Added for your viewing convenience is a continuous preview mode for the PDF. This mode is enabled by default, but can also be disabled through the \emph{(View $\rightarrow$ Page layout in preview)} menu. Complementary to this feature is SyncTeX integration, which allows you to synchronize the position in your editor with the PDF preview. 

\section{Feedback}
We hope you will enjoy using this release as much as we enjoyed creating it. If you have comments, suggestions or wish to report an issue you are experiencing - contact us at: \emph{http://gummi.midnightcoding.org}.

\section{One more thing}
If you are wondering where your old default text is; it has been stored as a template. The template menu can be used to access and restore it. 

\end{document}
